\newcommand{\R}{ℝ}
\newcommand{\N}{ℕ}
%\mathscr is rounder than \mathcal.
\newcommand{\powerset}[1]{\mathscr{P}(#1)}
%Powerset without zero.
\newcommand{\powersetz}[1]{\mathscr{P}^*(#1)}
%https://tex.stackexchange.com/a/45732, works within both \set and \set*, same spacing than \mid (https://tex.stackexchange.com/a/52905).
\newcommand{\suchthat}{\;\ifnum\currentgrouptype=16 \middle\fi|\;}
%Integer interval.
\newcommand{\intvl}[1]{⟦#1⟧}
%Allows for \abs and \abs*, which resizes the delimiters.
\DeclarePairedDelimiter\abs{\lvert}{\rvert}
\DeclarePairedDelimiter\card{\lvert}{\rvert}
\DeclarePairedDelimiter\norm{\lVert}{\rVert}
%Better than using the package braket because braket introduces possibly undesirable space. Then: \begin{equation}\set*{x \in \R^2 \suchthat \norm{x}<5}\end{equation}.
\DeclarePairedDelimiter\set{\{}{\}}
\DeclareMathOperator*{\argmax}{arg\,max}
\DeclareMathOperator*{\argmin}{arg\,min}

%We want the straight form of \phi for mathematics, as recommended in UTR #25: Unicode support for mathematics, and thus use \phi for the mathematical symbol and not \varphi; and similarly \epsilon is preferred to \varepsilon for the mathematical symbol.

%The amssymb solution.
%\newcommand{\restr}[2]{{#1}_{\restriction #2}}
%Another acceptable solution.
%\newcommand{\restr}[2]{{#1|}_{#2}}
%https://tex.stackexchange.com/a/278631
\newcommand\restr[2]{#1\raisebox{-.5ex}{$|$}_{#2}}


